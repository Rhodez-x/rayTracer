\documentclass[a4paper,10pt]{article}

\usepackage[utf8]{inputenc}
\usepackage[T1]{fontenc}
\usepackage[danish]{babel}

\usepackage{color}
\usepackage{float}
\usepackage{fancyvrb}

\usepackage{amssymb}
\usepackage{amsmath}
\usepackage{listings}
\usepackage{comment} 

\usepackage{graphicx}
\DeclareGraphicsExtensions{.png}

\definecolor{dkgreen}{rgb}{0,0.45,0}
\definecolor{gray}{rgb}{0.5,0.5,0.5}
\definecolor{mauve}{rgb}{0.30,0,0.30}

% Default settings for code listings
% Default settings for code listings
\lstset{frame=tb,
  language=Java,
  aboveskip=3mm,
  belowskip=3mm,
  showstringspaces=false,
  columns=flexible,
  basicstyle={\small\ttfamily},
  numbers=left,
  numberstyle=\footnotesize,
  keywordstyle=\color{dkgreen}\bfseries,
  commentstyle=\color{dkgreen},
  stringstyle=\color{mauve},
  frame=single,
  breaklines=true,
  breakatwhitespace=false
  tabsize=1
}

%indsætjeres navn herunder.
\title{førsteårs projekt\\\rule{10cm}{0.5mm}}
\author{Jørn Guldberg 
\\Username: jogul16
\\Supervisor: Simon Larsen\\ FF501\\\rule{5.5cm}{0.5mm}\\}
\date{\today}

\begin{document}

\maketitle



\newpage
\section{Abstract} 
%Resume af projekt
Ray-tracing is a method for rendering 3D computer graphics.​ Its render a 3d world into a 2d image. 
A ray tracer sends a ray for every pixel in the image, and calculate the color and light settings for the pixel.
Calculation is bassed on intersection with objects in the 3d world.. 

We are implementing a simple ray-tracer in Java and using Vector math for calculation of the world scene. ​
Our world contains Spheres, triangels and panes. These have an intersect method to check intersects.
The world is spilt up into volume boxes for faster ​time complexity.
We futhermore investigate how ray-tracing is different for other methods for rendering graphics and
the advances of useing voulume boxes. 

** RESULTS / CONCLUSION??**
\newpage
\vfill
\tableofcontents
%laver forside og indholsfortegnelse

\newpage
\section{Indledning}
% Indledning som inkludere problemformulering
Why does raytracing require a substantial amount of processing power and how efficiently can data structures Improve time complexity for ray-tracing algorithms ​
\newpage

\section{Teori}
% tjek med simon hvor meget der skal skrives her.

\newpage

\section{Materialer og metoder} 
% Igen tjek hvad og hvor meget der skal skrives her. 

\newpage

\section{Results} 
% Our results

\newpage

\section{Disscussion} 


\newpage

\section{Conclusion} 


\newpage

\section{Perspektivering} 


\newpage

\section{Literaturliste} 


\newpage

\section{Processevaluering} 


\newpage

\section{Appendix (source code)}
%Spørg om hvorvid vi skal have apenddix, og om sourcefiler mm. skal afleveres. 


\end{document}
